% !TeX root = ../whu-proposal-main.tex

\section{如何使用此模版}

此文件既是示例文件,也是模版的说明文档。本模版的使用方法如下:

\subsection{如何选择本科或硕博的版本}

在 \file{whu-proposal-main.tex} 的 \tn{documentclass} 命令中,type 选择 bachelor 或 master 或 doctor 选项,即可选择本科或硕博的版本。


\subsection{个人信息}

在 \file{whu-proposal-main.tex} 的 \tn{ProposalSetup} 命令中填写个人信息,包括姓名、学号、专业、研究方向、导师姓名、日期等信息。其中,题目、学院、学号和姓名为本科生必填信息,其他信息为研究生必填信息。


\subsection{正文内容}

个人推荐使用 \tn{input} 命令实现分章节编译,使得主文件(即本文件)结构更清晰(就像 \file{whu-proposal-main.tex} 文件的处理一样)。

章节是三级结构,即 \tn{section}、\tn{subsection} 和 \tn{subsubsection}。

如果不需要目录,可以将 \file{whu-proposal-main.tex} 文件的 \tn{tableofcontents} 命令注释掉。


\subsection{如何编译}

如果暂时不需要编译参考文献,请使用 \tool{xelatex} 编译。但要注意的是,由于硕博模版中的实现原理,没有报错的前提下,\emph{初次编译} 的时候可能需要 \emph{至少} 编译两次以上(基本上最多四次),才能正确生成封面、目录(如果有)和边框。


\subsection{参考文献}

模板已经提前配置好了 \pkg{biblatex} 宏包(见导言区)。如果你不会自己配置,只需要按照我下面的步骤来。

\subsubsection{如何引用参考文献}

请将参考文献信息填入 \file{whu-proposal.bib} 文件中,然后根据需要在正文中使用 \tn{cite} 或者 \tn{parencite} 命令引用参考文献。例如:
\begin{itemize}
  \item \tn{cite}:上标引用一篇文献\cite[定理 1]{研究生学位论文开题报告登记表}。

  \item \tn{parencite}:行内引用一篇文献 \parencite[定理 1]{研究生学位论文开题报告登记表}。
\end{itemize}


\subsubsection{参考文献的编译方式}

使用 \tool{biber} 编译参考文献,编译顺序为 \tool{xelatex} $\to$ \tool{biber} $\to$ \tool{xelatex} $\to$ \tool{xelatex},或者使用 \tool{latexmk -xelatex whu-proposal-main} 编译。