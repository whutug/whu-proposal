% !TeX root = ../whu-proposal-main.tex

\section{可供用户参考的内容}

\subsection{2023年硕博开题报告的时间安排}

(以下来自班级群班长转发内容\cite{2024届硕博开题报告QQ通知})

各位同学,学位论文开题报告会预计在下学期9月开展,请大家做好开题准备:
\begin{enumerate}
  \item 开题报告会时间:2023年9中下旬(暂定),确切日期等各系通知;
  \item 开题前与导师及时、充分沟通论文选题及研究计划,确保选题合理,研究计划可行;开题需征得导师同意;
  \item 请将开题报告初稿于7月31日前发给导师审核,根据导师意见修改和完善;
  \item 硕士研究生开题报告不少于3千字,博士研究生开题报告不少于5千字,开题报告内容包括论文选题意义、国内外研究现状、论文研究计划、主要参考问题。行文主要篇幅用于写“论文研究计划”;
  \item 参加此次开题报告的学生名单见下表,如有遗漏,请告知教学办;
  \item 开题是进行学位论文撰写的必经环节,请大家认真对待。
\end{enumerate}



\subsection{硕博开题报告的内容组成}

下面内容来自《2023 年 3 月学位论文开题报告的通知》\cite{2023届硕博学位论文开题报告通知}:

开题报告的内容一般由以下几部分组成:
\begin{enumerate}
  \item 论文选题的价值与意义;
  \item 与论文选题相关的国内外研究现状及趋势;
  \item 论文研究计划,包括
    \begin{enumerate}
      \item 研究目标、研究思路、研究方法、技术路线;
      \item 研究的重难点、可能的创新点与不足之处;
      \item 研究的主要内容及初步框架;
      \item 研究的进度安排;
    \end{enumerate}
  \item 主要参考文献。
\end{enumerate}

硕士研究生开题报告正文部分不得少于 3000 字,博士研究生开题报告正文部分不得少于 5000 字。