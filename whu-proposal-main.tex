% !TeX encoding = UTF-8
% !TeX program = xelatex
% !TeX spellcheck = en_US

%********************************************
% whu-proposal: 武汉大学开题报告模版

% Github:https://github.com/whutug/whu-proposal
% Gitee:https://gitee.com/xkwxdyy/whu-proposal
% Update date: 2023-09-26
% Version: v0.8
% Author: Kangwei Xia, kangweixia_xdyy@163.com, School of Mathematics and Statistics, Wuhan University
% QQ group: 681965476
%********************************************


% \documentclass[type = bachelor]{whu-proposal}  % 本科生
% \documentclass[type = master]{whu-proposal}      % 硕士生
\documentclass[type = doctor]{whu-proposal}      % 博士生


% 个人信息
\ProposalSetup{
  title            = { 论文题目 } ,
  department       = { 数学与统计学院 } ,
  student_id       = { 2021202012345 } ,
  author           = { 作者姓名 } ,
  % 下面的内容只有【硕|博】需要填写
  major            = { 基础数学 } ,
  research_area    = { 算子理论 } , 
  supervisor       = { 导师姓名 } ,
  supervisor_title = { 教授 } ,
  year             = { 2023 },  % 年份不填写时默认为编译时的年份
  % month            = { 5 },     % 月份不填写时默认为编译时的月份
  % day              = { 21 },    % 日期不填写时默认为编译时的日期
}


% 载入所需宏包,下面的仅为示例文件中需要,实际使用时可根据需要增减

% 参考文献
\usepackage[
  backend      = biber,
  bibstyle     = gb7714-2015,
  citestyle    = gb7714-2015,
  sorting      = nyt,
  gbnamefmt    = givenahead,
  gbpunctin    = false
]{biblatex}
\addbibresource{whu-proposal.bib}  % 参考文献数据库



% 自定义命令,下面的仅为示例文件中需要,实际使用时可根据需要增减
\newcommand\tool{\texttt}
\newcommand\tn[1]{\texttt{\textbackslash#1}}
\newcommand\file{\nolinkurl}
\newcommand\pkg{\textsf}



\begin{document}

% 目录,不需要的话可去除
% \tableofcontents


% 正文内容
% !TeX root = ../whu-proposal-main.tex

\section{为什么会开发此模版}

武汉大学本科生的开题报告模版已经\href{https://github.com/whutug/whu-thesis}{由 whu-tug 开发},但是研究生的模版仍然只有 word 版本,对于数学系的同学来说,相较于 \LaTeX{} 来说,使用 word 进行排版数学公式并不是一件容易的事情。而且 \LaTeX{} 使用 bib 数据库进行参考文献的管理,可以很方便地实现参考文献的引用。

于是我基于《附件1:研究生学位论文开题报告登记表》\cite{研究生学位论文开题报告登记表} 开发了此模版,希望能够让同学们更方便、高效地完成开题报告的撰写。

\emph{注意}:开题报告上传系统的时候可能只能上传 \texttt{.doc} 文件,有以下解决办法:
\begin{enumerate}
  \item 将 \LaTeX{} 编译生成的 PDF 转为图片(一般的 PDF 编辑器都可以做到)
  \item 将图片插入 \texttt{.doc} 或 \texttt{.docx} 文件中保存
  \item 上传系统
\end{enumerate}
% !TeX root = ../whu-proposal-main.tex

\section{如何使用此模版}

此文件既是示例文件,也是模版的说明文档。本模版的使用方法如下:

\subsection{如何选择本科或硕博的版本}

在 \file{whu-proposal-main.tex} 的 \tn{documentclass} 命令中,type 选择 bachelor 或 master 或 doctor 选项,即可选择本科或硕博的版本。


\subsection{个人信息}

在 \file{whu-proposal-main.tex} 的 \tn{ProposalSetup} 命令中填写个人信息,包括姓名、学号、专业、研究方向、导师姓名、日期等信息。其中,题目、学院、学号和姓名为本科生必填信息,其他信息为研究生必填信息。


\subsection{正文内容}

个人推荐使用 \tn{input} 命令实现分章节编译,使得主文件(即本文件)结构更清晰(就像 \file{whu-proposal-main.tex} 文件的处理一样)。

章节是三级结构,即 \tn{section}、\tn{subsection} 和 \tn{subsubsection}。

如果不需要目录,可以将 \file{whu-proposal-main.tex} 文件的 \tn{tableofcontents} 命令注释掉。


\subsection{如何编译}

如果暂时不需要编译参考文献,请使用 \tool{xelatex} 编译。但要注意的是,由于硕博模版中的实现原理,没有报错的前提下,\emph{初次编译} 的时候可能需要 \emph{至少} 编译两次以上(基本上最多四次),才能正确生成封面、目录(如果有)和边框。


\subsection{参考文献}

\subsubsection{如何引用参考文献}

上标引用一篇文献\cite{研究生学位论文开题报告登记表}。

行内引用一篇文献\parencite{研究生学位论文开题报告登记表}。


\subsubsection{参考文献的编译方式}

使用 \tool{biber} 编译参考文献,编译顺序为 \tool{xelatex} $\to$ \tool{biber} $\to$ \tool{xelatex} $\to$ \tool{xelatex},或者使用 \tool{latexmk -xelatex whu-proposal-main} 编译。
% !TeX root = ../whu-proposal-main.tex

\section{可供用户参考的内容}

\subsection{2023年硕博开题报告的时间安排}

(以下来自班级群班长转发内容\cite{2024届硕博开题报告QQ通知})

各位同学,学位论文开题报告会预计在下学期9月开展,请大家做好开题准备:
\begin{enumerate}
  \item 开题报告会时间:2023年9中下旬(暂定),确切日期等各系通知;
  \item 开题前与导师及时、充分沟通论文选题及研究计划,确保选题合理,研究计划可行;开题需征得导师同意;
  \item 请将开题报告初稿于7月31日前发给导师审核,根据导师意见修改和完善;
  \item 硕士研究生开题报告不少于3千字,博士研究生开题报告不少于5千字,开题报告内容包括论文选题意义、国内外研究现状、论文研究计划、主要参考问题。行文主要篇幅用于写“论文研究计划”;
  \item 参加此次开题报告的学生名单见下表,如有遗漏,请告知教学办;
  \item 开题是进行学位论文撰写的必经环节,请大家认真对待。
\end{enumerate}



\subsection{硕博开题报告的内容组成}

下面内容来自《2023 年 3 月学位论文开题报告的通知》\cite{2023届硕博学位论文开题报告通知}:

开题报告的内容一般由以下几部分组成:
\begin{enumerate}
  \item 论文选题的价值与意义;
  \item 与论文选题相关的国内外研究现状及趋势;
  \item 论文研究计划,包括
    \begin{enumerate}
      \item 研究目标、研究思路、研究方法、技术路线;
      \item 研究的重难点、可能的创新点与不足之处;
      \item 研究的主要内容及初步框架;
      \item 研究的进度安排;
    \end{enumerate}
  \item 主要参考文献。
\end{enumerate}

硕士研究生开题报告正文部分不得少于 3000 字,博士研究生开题报告正文部分不得少于 5000 字。



% 参考文献

% \printbibliography  % “参考文献”一节标题不编号
\printbibliography[heading = bibnumbered]  % “参考文献”一节标题编号


\end{document}